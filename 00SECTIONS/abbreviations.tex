%% -*- Mode: LaTeX -*-
%%
%% abbreviations.txt
%% Created Fri May 31 10:46:19 AKDT 2019
%% Copyright (C) 2019 by Raymond E. Marcil <marcilr@gmail.com>
%%

%%
%% Use of hyperref \href:
%%   \href{URL}{DESCRIPTION} 
%%

%% =============== List of Abbreviations ===============
%% =============== List of Abbreviations ===============
\newpage
\setcounter{secnumdepth}{0}
\section{List of Definitions and Abbreviations}

\noindent\begin{itemize*}

%%
%% ===================== CONTAINER DATABASE ===================
%% ===================== CONTAINER DATABASE ===================
%%
\item{\begin{bf}Container database\end{bf}} - ``Container Database (CDB) : On 
the surface this seems very similar to a conventional Oracle database, as it 
contains most of the working parts you will be already familiar with 
(controlfiles, datafiles, undo, tempfiles, redo logs etc.). It also houses 
the data dictionary for those objects that are owned by the root container 
and those that are visible to all PDBs.''\footnote{Multitenant : Overview of Container Databases (CDB) and Pluggable Databases (PDB)\\
\href{https://oracle-base.com/articles/12c/multitenant-overview-container-database-cdb-12cr1\#create-pdb}{https://oracle-base.com/articles/12c/multitenant-overview-container-database-cdb-12cr1\#create-pdb}}



%%
%% ========================== DATABASE ========================
%% ========================== DATABASE ========================
%%
\item{\begin{bf}Database\end{bf}} - ``A database is a set of physical
  files on disk created by the CREATE DATABASE statement. The instance
  manages its associated data and serves the users of the database.''\footnote{Introduction to the Oracle Database Instance, Oracle Database Instance, Database Concepts, Oracle Database,
    Release 12.2,
    \href{https://docs.oracle.com/en/database/oracle/oracle-database/12.2/cncpt/oracle-database-instance.html}{https://docs.oracle.com/en/database/oracle/oracle-database/12.2/cncpt/}\newline
  \href{http://www.google.com}{oracle-database-instance.html}}

%%
%% ============================ DBCA ==========================
%% ============================ DBCA ==========================
%%
\item{\begin{bf}DBCA\end{bf}} - Database Configuration Assistant.  The DBCA can be used to:\footnote{\href{https://docs.oracle.com/cd/B16254_01/doc/server.102/b14196/install003.htm}{Using DBCA to Create and Configure a Database, Oracle\textregistered \hspace{1pt} Database 2 Day DBA, }}
\begin{itemize*}
  \item Creating a Database with DBCA
  \item Configuring Database Options with DBCA
  \item Deleting a Database with DBCA
  \item Managing Templates with DBCA
  \item Configuring Automatic Storage Management with DBCA
\end{itemize*}


%%
%% ========================== INSTANCE ========================
%% ========================== INSTANCE ========================
%%
\item{\begin{bf}Instance\end{bf}} - ``A database instance
  is a set of memory structures that manage database files.''\footnote{Ibid.}
  

%%
%% ========================= MULTITENANT ======================
%% ========================= MULTITENANT ======================
%%
\item{\begin{bf}Multitenant\end{bf}} - ``The multitenant option
  introduced in Oracle Database 12c allows a single container
  database (CDB) to host multiple separate pluggable databases
  (PDB).''\footnote{Multitenant : Create and Configure a Pluggable Database (PDB) in Oracle Database 12c Release 1 (12.1)\\
\href{https://oracle-base.com/articles/12c/multitenant-create-and-configure-pluggable-database-12cr1}{https://oracle-base.com/articles/12c/multitenant-create-and-configure-pluggable-database-12cr1}}

%%
%% =============== ORACLE UNIVERSAL INSTALLER (OUI) ===========
%% =============== ORACLE UNIVERSAL INSTALLER (OUI) =========== 
%%
\item{\begin{bf}Oracle Universal Installer (OUI)\end{bf}} - 'The Oracle
  Universal Installer (OUI) allows you to create a container database (CDB)
  during the software installation. The "Typical Install Configuration"
  screen has a checkbox to indicate the database is a container database.
  You can optionally create a single pluggable database (PDB) in this
  screen also.'\footnote{Oracle Universal Installer (OUI)\\
    \href{https://oracle-base.com/articles/12c/multitenant-create-and-configure-container-database-12cr1\#oui}{https://oracle-base.com/articles/12c/multitenant-create-and-configure-container-database-12cr1\#oui}}

%%
%% ===================== PLUGGABLE DATABASE ===================
%% ===================== PLUGGABLE DATABASE ===================
%%
\item{\begin{bf}Pluggable database\end{bf}} - ``Pluggable Database (PDB) : Since 
the CDB contains most of the working parts for the database, the PDB only n
eeds to contain information specific to itself. It does not need to worry about 
controlfiles, redo logs and undo etc. Instead it is just made up of datafiles 
and tempfiles to handle it's own objects. This includes it's own data dictionary, 
containing information about only those objects that are specific to the PDB. 
From Oracle 12.2 onward a PDB can, and should, have a local undo tablespace.''\footnote{Multitenant : Overview of Container Databases (CDB) and Pluggable Databases (PDB)\\
\href{https://oracle-base.com/articles/12c/multitenant-overview-container-database-cdb-12cr1\#create-pdb}{https://oracle-base.com/articles/12c/multitenant-overview-container-database-cdb-12cr1\#create-pdb}}




\end{itemize*}


%%\footnote{What is AWStats, \href{https://awstats.sourceforge.io/\#WHAT}{https://awstats.sourceforge.io/\#WHAT}}}
%%
%%However if you use AWStats as a CGI you can click on the ``year'' link to
%%have a report for all the year. In such a report, period is a full year,
%%so Unique Visitors are number of hosts that have made at least 1 hit on
%%1 page of your web site during the year.


%%
%% Need to complete with items from:
%% AWStats Glossary
%% https://awstats.sourceforge.io/docs/awstats_glossary.html
%%
